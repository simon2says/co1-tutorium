\documentclass{scrartcl} % file is in UTF-8 Encoding

\usepackage[T1]{fontenc}

\addtolength{\textwidth}{2cm}
\addtolength{\oddsidemargin}{-1,3cm}
\addtolength{\topmargin}{-2.5cm}
\addtolength{\textheight}{6cm}
\addtolength{\footskip}{1cm}
\typearea{13}

\marginparwidth=1.0in
\setlength{\parindent}{0pt}
\pagestyle{empty}


%%%%%%%%%%%%%%%%%%%%%%%%%%%%%%%%%%%%% Packages %%%%%%%%%%%%%%%%%%%%%%%%%%%%%%%%%

\usepackage[ngerman]{babel}
\usepackage{hyperref}
\usepackage{ae,aecompl}         % für gute PDF-Fonts
\usepackage{amssymb}
\usepackage{amsmath}
\usepackage{amsthm}
\usepackage{enumerate}
\usepackage{caption}
\usepackage[ruled, linesnumbered, noline]{algorithm2e}
\usepackage{tabularx}
\usepackage{xspace}
\usepackage{listings}
\usepackage{color}
\usepackage{tikz}
\usepackage{enumitem}
\usepackage{fancyhdr}
\usepackage{lastpage}


%%%%%%%%%%%%%%%%%%%%%%%%%%%%%%%%%%%%% Own Commands %%%%%%%%%%%%%%%%%%%%%%%%%%%%%
\DeclareMathOperator{\N}{\mathbb N}
\DeclareMathOperator{\Q}{\mathbb Q}
\DeclareMathOperator{\R}{\mathbb R}
\DeclareMathOperator{\Z}{\mathbb Z}

% Unser Kopf
\newcommand{\university}{Technische Universität Berlin}
\newcommand{\semester}{WiSe 2024/2025}
\newcommand{\faculty}{Fakultät II, Institut für Mathematik}
\newcommand{\head}
{
	\noindent \textbf{\university} \hfill
	\textbf{\semester} \\
	\textbf{\faculty} \\
}

\newcommand{\Bemerkung}[1]{\par\noindent \textbf{Bemerkung:} #1 \xspace}
\newtheorem{theorem}{Satz}
\newtheorem*{theorem*}{Satz}
\newtheorem{definition}{Definition}
\newtheorem*{definition*}{Definition}
\theoremstyle{remark}
\newtheorem{claim}[theorem]{Behauptung}

\newcommand{\student}[2]{#1 & #2\\}  
\newcommand{\group}[1]{\underline{Gruppe #1}}


% Farbschema fuer Listings


\definecolor{codegreen}{rgb}{0,0.6,0}
\definecolor{codegray}{rgb}{0.5,0.5,0.5}
\definecolor{codepurple}{rgb}{0.58,0,0.82}
\definecolor{backcolour}{rgb}{0.95,0.95,0.92}

\hypersetup{linkbordercolor=backcolour}

\lstdefinestyle{mystyle}{
	backgroundcolor=\color{backcolour},   
	commentstyle=\color{codegreen},
	keywordstyle=\color{magenta},
	numberstyle=\tiny\color{codegray},
	stringstyle=\color{codepurple},
	basicstyle=\ttfamily\footnotesize,
	breakatwhitespace=false,         
	breaklines=true,                 
	captionpos=b,                    
	keepspaces=true,                 
	numbers=left,                    
	numbersep=5pt,                  
	showspaces=false,                
	showstringspaces=false,
	showtabs=false,                  
	tabsize=2
}

\lstset{style=mystyle}

%%%%%%%%%%%%%%%%%%%%%%%%%%%%%
\newcounter{auf}\setcounter{auf}{0}
\newenvironment{aufgabe}
{
	\addvspace{\medskipamount}
	\refstepcounter{auf}
	\noindent
	\textbf{\arabic{auf}. Tutoriumsaufgabe}
	\nopagebreak{\medskip \par}
}

\newcommand{\haloesung}[1]{
	\begin{center}
		{\large \bfseries Lösungen zum #1.\ Tutoriumsblatt Computerorientierte Mathematik I}
		\vskip1ex
	\end{center}
}

\pagestyle{fancy}
\fancyhf{}
\fancyfoot[C]{-- \thepage\ of \pageref{LastPage} --}
\fancypagestyle{startseite}{
	\fancyhead[L]{\includegraphics[width=1cm]{C:/Users/chole/Documents/tub/tublogo.png}}
	\fancyhead[R]{{\bfseries\university\\ \faculty}}
	\fancyfoot[C]{-- \thepage\ of \pageref{LastPage} --}
}
\setlength{\headheight}{33pt}
\thispagestyle{startseite}

%---------------- VARS --------------%
\newcommand{\headvorlage}{	
	\begin{tabular}{@{}l r}
		\student{Simon Seemüller}{513084}
	\end{tabular}\hfill \textbf{\semester} \\
	\haloesung{3}
}

\begin{document}
	\headvorlage
	\begin{aufgabe}
		\begin{enumerate}[label=(\alph*)]
			\item \begin{itemize}
				\item $(1101)_2 = 13$
				\item $25 = (11001)_2$
				\item $789 = (11124)_5$
 			\end{itemize}
			\item Die größte Zahl, die in $\ell$ Ziffern in $b$-adischer Darstellung dargestellt werden kann, ist $b^\ell - 1$.
			\item Die Länge von $z$ in $b$-adischer Darstellung beträgt $\ell = \lfloor \log_b(z) \rfloor + 1.$
			\item $(0110111100110101)_{2} = (6F35)_{16}$
			\item \begin{itemize}
				\item $(1101)_2 + (110)_2 = (10011)_2$
				\item $(1101)_2 - (110)_2 = (111)_2$
				\item $(1101)_2 \cdot (110)_2 = (1001110)$
				\item $(1101)_2 \div (110)_2 = (1.0\overline{01})_2$
			\end{itemize}
		\end{enumerate}
	\end{aufgabe}
	\begin{aufgabe}
		\begin{enumerate}[label=(\alph*)]
			\item $K_2^8(01100110) = 10011010$
			\item $-5$ wird in das Zweierkomplement mit Länge $4$ übersetzt, indem fünf von der Zahl $10000$ subtrahiert wird. Man erhält so $10000 - 101 = 1011$. Wenn stattdessen acht Stellen zur Verfügung stehen, werden die restlichen vorne mit $1$ aufgefüllt.
			\item Die Zweierkomplementdarstellung von $a29c1f$ ist $5d63e1$, da $1000000 - a29c1f = 5d63e1$. $n$ ist negativ, da die Zahl sich in der oberen Hälfte des Zahlenbereichs befindet.
			\item Die Zahl $11001100$ in Zweierkomplement beschreibt die Zahl $-52$. Sie befindet sich in der oberen Hälfte des Zahlenbereiches, ist also negativ, daher nehmen wir das Ergebnis der Subtraktion.
			\item Die beiden Zahlen können einfach zusammenaddiert werden und ergeben dann $2212$, die Zweierkomplementdarstellung für das Ergebnis der Addition von $17$ und $-21$, also $-4$.
			\item $10 - 15$ im Zweierkomplement sind $10 = (1010)_2$ und $-15 = (10001)_2$. Die Addition der beiden ergibt $(1010)_2 + (10001)_2 = (11011)_2$, was das Zweierkomplement für $-5$ ist. Falls $-2-15$ gerechnet wird, ist das Ergebnis unterhalb der erlaubten unteren Bereichsgrenze von $-\lfloor \frac{b^l}{2}\rfloor$
		\end{enumerate}
	\end{aufgabe}
	\newpage
	\begin{aufgabe}
		\begin{enumerate}[label=(\alph*)]
			\item\hspace*{1ex}\\[-2ex]
			\lstinputlisting[language=Python, firstline=1,lastline=4,caption={Berechnung des Maximums zweier Zahlen}]{./python/maximum.py}%
			\item\hspace*{1ex}\\[-2ex]
			\lstinputlisting[language=Python, firstline = 6, caption={Berechnung des Maximums dreier Zahlen}]{./python/maximum.py}%
			\item\hspace*{1ex}\\[-2ex]
			\lstinputlisting[language=Python, caption={Berechnung der Fakultät einer Zahl $n$}]{./python/fac.py}
			\item\hspace*{1ex}\\[-2ex]
			\lstinputlisting[language=Python, caption={Berechnung der Summe der ersten $n$ geraden Zahlen, Variante 1},lastline=5]{./python/sum.py}
			\lstinputlisting[language=Python, caption={Berechnung der Summe der ersten $n$ geraden Zahlen, Variante 2},firstline=7,lastline=16]{./python/sum.py}
			\newpage
			\item\hspace*{1ex}\\[-2ex]
			\lstinputlisting[language=Python, caption={Berechnung der Collatz-Folge einer Zahl $n$}]{./python/collatz.py}
		\end{enumerate}
	\end{aufgabe}
	\begin{aufgabe}
		\begin{enumerate}[label=(\alph*)]
			\item zu zeigen:
			\[
				\forall x = \sum_{i = -N}^{M}x_ib^i \exists p, q: x = \frac{p}{q}
			\]
			Wähle hierfür $q = b^{N}$ und $p = \sum_{i = -N}^{M}x_ib^{i+N}$. Dann ist
			\[
				x = \frac{\sum_{i = -N}^{M}x_ib^{i + N}}{b^{N}} = \sum_{i = -N}^{M}x_i b^i
			\]
			\item Wähle $x = \frac{1}{5}$ und $b = 6$. Dann ist $(x)_b = 0,\overline{1}$
			\item 
			\item 
		\end{enumerate}
	\end{aufgabe}
\end{document}